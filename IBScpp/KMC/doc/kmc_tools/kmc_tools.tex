\documentclass[a4paper,11pt]{article}

\usepackage[T1]{fontenc}
\usepackage{booktabs}
\usepackage{url}
\usepackage{palatino}
\usepackage{graphicx}
\usepackage{float}
\usepackage{listings}
\usepackage{color}
\usepackage[hidelinks]{hyperref}


\usepackage{xcolor}
\hypersetup{
	colorlinks,
	linkcolor={blue!80!black},
	citecolor={blue!80!black},
	urlcolor={blue!80!black}
}

\usepackage{cleveref}
%% \usepackage{tikz}
%% \usetikzlibrary{arrows}

%\usepackage{rotating}

\urlstyle{sf}

\setlength\oddsidemargin{-0.5in}
\setlength\evensidemargin{-0.5in}

\setlength\textheight{\paperheight}
\addtolength\textheight{-\topmargin}
\addtolength\textheight{-\headheight}
\addtolength\textheight{-\headsep}
\addtolength\textheight{-2in}
\setlength\textwidth{\paperwidth}
\addtolength\textwidth{-1in}

\setlength\topmargin{0in} 
\emergencystretch=5pt 

% ***********************************************************
\lstset{language=C++}
\lstset{basicstyle=\scriptsize\sffamily}
\lstset{keywordstyle=\bfseries}
\lstset{morekeywords={__shared__,__global__,uint2,uint4}}
\lstset{numbers=left}
\lstset{numberstyle=\sffamily\tiny}
\lstset{tabsize=2}


\newcommand{\q}{\phantom0}
\newcommand{\qq}{\phantom{00}}
\newcommand{\qqq}{\phantom{000}}
\newcommand{\qqqq}{\phantom{0000}}
\newcommand{\qqqqq}{\phantom{00000}}
\newcommand{\qc}{\phantom{0,}}
\newcommand{\qcq}{\phantom{0,0}}
\newcommand{\qcqq}{\phantom{0,00}}
\newcommand{\qcqqq}{\phantom{0,000}}
\newcommand{\qcqqqc}{\phantom{0,000,}}
\newcommand{\qcqqqcq}{\phantom{0,000,0}}
\newcommand{\qqc}{\phantom{00,}}
\newcommand{\GDCra}[2]{GDC-ra-$2^{#1}$-$2^{#2}$}
\newcommand{\qa}{\phantom{$^*$}}

\newcommand{\mcOT}{\multicolumn{3}{c}{\em out of time ($>10$ hours)}}
\newcommand{\mcOM}{\multicolumn{3}{c}{\em out of memory}}
\newcommand{\mcOD}{\multicolumn{3}{c}{\em out of disk ($>650$\,GB)}}
\newcommand{\mcNS}{\multicolumn{3}{c}{\em unsupported $k$}}
\newcommand{\mcF}{\multicolumn{3}{c}{\em failed}}

\newcommand{\comment}[1]{{\color{red} #1}}

\newcommand{\algname}[1]{\textsc{#1}}
\newcommand{\algdef}[1]{\textsc{#1}\label{algdef:#1}}

\newcounter{lnoc}
\newenvironment{algorithm}[1]{%
\hrule height 0.8pt \vspace{0.6ex} \small#1\vspace{0.6ex}\hrule height 0.5pt \vspace{-2.25ex}
\setcounter{lnoc}{0}
\small
\begin{tabbing}
00000\=XX\=XX\=XX\=XX\=XX\=XX\=\kill
}{%
\end{tabbing}
\vspace{-2.5ex}\hrule height 0.8pt\vspace{1ex}}
\newcommand{\lno}[1][0]{{\footnotesize\sffamily 
\ifnum#1=0
\stepcounter{lnoc} 
\ifnum\thelnoc<10
\phantom0%
\fi
\thelnoc
\else
\thelnoc.#1
\fi
}\>}

\newcommand{\pcfor}{{\bfseries for~}}
\newcommand{\pcforall}{{\bfseries for all~}}
\newcommand{\pcforeach}{{\bfseries for each~}}
\newcommand{\pcto}{{\bfseries to~}}
\newcommand{\pcdownto}{{\bfseries downto~}}
\newcommand{\pcdo}{{\bfseries do~}}
\newcommand{\pcif}{{\bfseries if~}}
\newcommand{\pcthen}{{\bfseries then~}}
\newcommand{\pcelse}{{\bfseries else~}}
\newcommand{\pcelseif}{{\bfseries else if~}}
\newcommand{\pcwhile}{{\bfseries while~}}
\newcommand{\pcbreak}{{\bfseries break}}
\newcommand{\pcand}{{\bfseries and~}}
\newcommand{\pcor}{{\bfseries or~}}
\newcommand{\pcnot}{{\bfseries not~}}
\newcommand{\pcreturn}{{\bfseries return~}}
\newcommand{\pcphreturn}{\phantom{\pcreturn}}
\newcommand{\pcfun}[1]{\mbox{\textrm{#1}}}
\newcommand{\pccomment}[1]{\qqq\{\textit{#1}\}}
\newcommand{\pctrue}{\mbox{\textrm{\bfseries true}}}
\newcommand{\pcfalse}{\mbox{\textrm{\bfseries false}}}
\newcommand{\pcdiv}{\mbox{\bfseries div~}}
\newcommand{\pcmod}{\mbox{\bfseries mod~}}
\newcommand{\pcinput}[1]{Input: \textit{#1}}
\newcommand{\pcinpute}[1]{\phantom{Input:} \textit{#1}}
\newcommand{\pcoutput}[1]{Output: \textit{#1}}
\newcommand{\pcoutpute}[1]{\phantom{Output:} \textit{#1}}
\newcommand{\pcendparam}{%
\end{tabbing}
\vspace{-0.7cm}
\hrule height 0.5pt
\vspace{-0.3cm}
\begin{tabbing}
00000\=XX\=XX\=XX\=XX\=XX\=XX\=\kill
}

\begin{document}

\centerline{\bfseries\LARGE\itshape kmc\_tools documentation}
\bigskip
\bigskip
\centerline{\bfseries\Large for v. 3.0.0}
\bigskip
\bigskip
\bigskip
\bigskip

\tableofcontents

\clearpage
\section*{Introduction}\label{sec:intro}
\addcontentsline{toc}{section}{\protect\numberline{}Introduction}
This document contains description of \textsf{kmc\_tools} software. \textsf{kmc\_tools} is a program which allows to work easily with sets of $k$-mers and their counters generated as output of \textsf{KMC}. \textsf{KMC} is an efficient $k$-mer counter described in: \url{http://bioinformatics.oxfordjournals.org/content/early/2015/02/18/bioinformatics.btv022}. \textsf{kmc\_tools} can work with databases produced by \textsf{KMC2} as well as by \textsf{KMC1} (there is a little difference between those formats). \textsf{kmc\_tools} always generates results in \textsf{KMC1} database format as it is a little faster (in sense of searching $k$-mers) than in case of \textsf{KMC2} database.

\clearpage
\section{kmc\_tools usage}
\label{sec:usage}

\textsf{kmc\_tools} provides a number of operations that can be used to work with $k$-mer sets. The number of input and output sets is depended on operation itself. Configuration of \textsf{kmc\_tools} is done via command-line parameters. \\
The general syntax is: \\
kmc\_tools [global\_params] $<$operation$>$ <operation\_params> \\ 

Global parameters are independent of operation type. There are:
\begin{itemize}
	\item \textsf{-t$<$value$>$} --- total number of threads (default: no. of CPU cores),
	\item \textsf{-v} --- enable verbose mode (shows some information) (default: false),
	\item \textsf{-hp} --- hide percentage progress (default: false).
\end{itemize}

Avaiable operations:

\begin{itemize}
	\item \textsf{simple} --- simple operations for two input sets (can produce multiple output sets),
	\item \textsf{complex} --- complex set operations for 2 or more input $k$-mer sets (can produce one output set),
	\item \textsf{transform} --- transform single $k$-mers set to other format (\textsf{KMC} database or text file, can produce multiple output sets),
	\item \textsf{filter} --- filter out reads with too small number of $k$-mers.
\end{itemize}
\textsf{simple} performs typical set operations with two input sets and may produce many output sets (e.g. one output for intersection and another one for union). This operation is described in \hyperref[sec:simple]{next section}. \textsf{complex} is able to perform user defined set operations with many inputs (see \hyperref[sec:complex]{section \ref{sec:complex}}). \textsf{transform} operation converts single input $k$-mer set to another. This operation allows to multiple convertions (resulting multiple output files). For details see \hyperref[sec:transform]{section \ref{sec:transform}}. The last operation takes as input \textsf{KMC} database and set of reads (e.g. FASTQ files) and keep only reads that contains at least specified number of $k$-mers (see \hyperref[sec:filter_operation]{section \ref{sec:filter_operation}}).


%\input{set_operations.tex}

\clearpage
\section{simple operation}
\label{sec:simple}
Command-line syntax: \\
%Syntax of \textbf{simple} operation: \\
kmc\_tools [global\_params] simple $<$input1 [input1\_params]$>$ $<$input2 [input2\_params]$>$ $<$oper1 output1 [output\_params1]$>$ [$<$oper2 output2 [output\_params2]$>$ ...] \\


where: 
\begin{itemize}
	\item \textsf{input1, input2} --- paths to the databases generated by \textsf{KMC} (\textsf{KMC} generates 2 files with the same name, but different extensions --- here only name without extension should be given),
	\item \textsf{oper1, oper2, ...} --- set operations to be performed on input1 and input2,
	\item \textsf{output1, output2, ...} --- paths of the output databases.
\end{itemize}

For each \textsf{input} there are additional parameters which can be set:
\begin{itemize}
	\item \textsf{-ci$<$value$>$} --- exclude $k$-mers occurring less than <value> times,
	\item \textsf{-cx$<$value$>$} --- exclude $k$-mers occurring more of than <value> times.
\end{itemize}

If additional parameters are not given they are taken from the appropriate input database. 

For each output there are also additional parameters:
\begin{itemize}
	\item \textsf{-ci$<$value$>$} --- exclude $k$-mers occurring less than <value> times,
	\item \textsf{-cx$<$value$>$} --- exclude $k$-mers occurring more than <value> times,
	\item \textsf{-cs$<$value$>$} --- maximal value of a counter,
	\item \textsf{-oc$<$value$>$} --- redefine counter calculation mode for equal $k$-mers.
	Available values:
	\begin{itemize}
		\item \textsf{min} --- get lower value of a $k$-mer counter,
		\item \textsf{max} --- get upper value of a $k$-mer counter,
		\item \textsf{sum} --- get sum of counters from both databases,
		\item \textsf{diff}  --- get difference between counters,
		\item \textsf{left} --- get counter from first database (input1),
		\item \textsf{right} --- get counter from second database (input2)
	\end{itemize}
\end{itemize}

If parameters are not given they are deduced based on input databases and specified operation.


Valid values for \textsf{oper1, oper2, ...} are:
\begin{itemize}
	\item \textsf{intersect} --- output database will contains only $k$-mers that are present in \textbf{both} input sets,
	\item \textsf{union} --- output database will contains each $k$-mer present in \textbf{any} of input sets,
	\item \textsf{kmers\_subtract} --- difference of input sets based on k-mers. Output database will contains only $k$-mers that are \textbf{present in the first} input set but \textbf{absent in the second} one,
	\item \textsf{counters\_subtract} --- difference of input sets based on k-mers and their counters (weaker version of kmers\_subtract). Output database will contains all $k$-mers that are present in the first input, without those for which counter operation will lead to remove such $k$-mer (i.e. counter equal to 0 or negative number),
	\item \textsf{reverse\_kmers\_subtract} --- same as kmers\_subtract but treat input2 as first and input1 as second,
	\item \textsf{reverse\_counters\_subtract} --- same as counters\_subtract but treat input2 as first and input1 as second.	
\end{itemize}
Each operation may be specified multiple times (which may be useful to produce two output sets with different cutoffs or counter calculation modes).



\begin{table}[H]
\centering
\begin{tabular}{|l|l|}
\hline
\textbf{operation} & \textbf{counter calculation mode} \\
\hline
\textsf{intersect} & \textsf{min} \\
\hline
\textsf{union} & \textsf{sum} \\
\hline
\textsf{kmers\_subtract} & NONE \\
\hline
\textsf{reverse\_kmers\_subtract} & NONE \\
\hline
\textsf{counters\_subtract} & \textsf{diff} \\
\hline
\textsf{reverse\_counters\_subtract} & \textsf{diff} (but input1 and input2 are swapped) \\
\hline
\end{tabular}
\caption{Default values of -oc switch for each operation}
\label{table:defaults_counter_mode}
\end{table}

For \textsf{kmers\_subtract} and \textsf{reverse\_kmers\_subtract} equal $k$-mers will never be present in the output database, which is the reason of NONE values in \cref{table:defaults_counter_mode}.

\subsection*{example 1}
kmc -k28 file1.fastq kmers1 tmp \\
kmc -k28 file2.fastq kmers2 tmp \\
kmc\_tools simple kmers1 -ci10 -cx200 kmers2 -ci4 -cx100 intersect kmers1\_kmers2\_intersect -ci20 -cx150 \\

\subsection*{example 2}
kmc -k28 file1.fastq kmers1 tmp \\
kmc -k28 file2.fastq kmers2 tmp \\
kmc\_tools simple kmers1 kmers2 intersect inter\_k1\_k2\_max -ocmax intersect inter\_k1\_k2\_min union union\_k1\_k2 -ci10 


\clearpage
\section{complex operation}
\label{sec:complex}

Complex operation allows to define operations for more than 2 input $k$-mer sets. \\
Command-line syntax: \\
kmc\_tools [global\_params] complex $<$operations\_definition\_file$>$ \\

where \textsf{operations\_definition\_file} is a path to the file which defines input sets and operations. It is a text file with the following syntax: \\ \\
\fbox{
	\parbox{\textwidth}{
		INPUT:\\
		$<$input1$>$ = $<$input1\_db\_path$>$ [params]\\
		$<$input2$>$ = $<$input2\_db\_path$>$ [params]\\
		...\\
		$<$inputN$>$ = $<$inputN\_db\_path$>$ [params]\\
		OUTPUT:\\
		$<$out\_db\_path$>$ = $<$ref\_input$>$ $<$oper [c\_mode]$>$ $<$ref\_input$>$ [$<$oper[c\_mode]$>$ $<$ref\_input$>$ [...]] \\
		\textbf{[}OUTPUT\_PARAMS: \\
		$<$output\_params$>$]
		
	}
}\\ 

where: 
\begin{itemize}
	\item \textsf{input1, input2, ..., inputN} --- names of inputs used to define operation,
	\item \textsf{input1\_db\_path, input2\_db\_path, inputN\_db\_path} --- paths of $k$-mer sets,
	\item \textsf{out\_db\_path} --- path of the output database,
	\item \textsf{ref\_input} is one of \textsf{input1, input2, ..., inputN}, 
	\item \textsf{oper} is one of \textsf{\{*,-,$\sim$,+\}}, the meaning is as follows:
	\begin{itemize}
		\item[\tiny$\bullet$] \textsf{*} --- intersect,
		\item[\tiny$\bullet$] \textsf{-} --- kmers\_subtract,
		\item[\tiny$\bullet$] \textsf{$\sim$} --- counters\_subtract,
		\item[\tiny$\bullet$] \textsf{+} --- union.
	\end{itemize}
	\item \textsf{c\_mode} --- redefine default counter calculation mode (available values: min, max, diff, sum, left, right).
\end{itemize}

For detailed description about operations and counter calculation mode see \hyperref[sec:simple]{section \ref{sec:simple}}

For each input there are additional parameters which can be set:
\begin{itemize}
	\item \textsf{-ci<value>} --- exclude $k$-mers occurring less than <value> times,
	\item \textsf{-cx<value>} --- exclude $k$-mers occurring more than <value> times.	
\end{itemize}

If additional parameters are not given they are taken from the appropriate input database. 
Operator \textsf{*} has the highest priority. Other operators has equal priorities. Order of operations can be changed with parentheses.

Available \textsf{output\_params}:
\begin{itemize}
	\item \textsf{-ci$<$value$>$} --- exclude $k$-mers occurring less than <value> times,
	\item \textsf{-cx$<$value$>$} --- exclude $k$-mers occurring more than <value> times,
	\item \textsf{-cs$<$value$>$} --- maximal value of a counter.
\end{itemize}
If the \textsf{output\_params} are not specified they are deduced based on input parameters.
\subsection*{example}
\fbox{
	\parbox{\textwidth}{
		INPUT:\\
		set1 = kmc\_o1 -ci5\\
		set2 = kmc\_o2\\
		set3 = kmc\_o3 -ci10 -cx100\\
		OUTPUT:\\
		result = (set3 + min set1) * set2 \\	
		OUTPUT\_PARAMS: \\
		-ci4 -cx80 -cs1000 \\
	}
}\\ \\


%\input{one_input_operations}

\clearpage
\section{transform}
\label{sec:transform}

This operation transforms single \textsf{KMC} database to one or more \textsf{KMC} database(s) or text file(s). \\
Command-line syntax: \\
kmc\_tools [global\_params] transform $<$input$>$ [input\_params] $<$oper1 [oper\_params1] output1 [output\_params1]$>$ [$<$oper2 [oper\_params2] output2 [output\_params2]$>$...]

where:


\begin{itemize}
	\item \textsf{oper1, oper2, ...} --- transform operation to be performed on the input,
	\item \textsf{input} -- path to databases generated by \textsf{KMC} (\textsf{KMC} generates 2 files with the same name, but different extensions --- here only name without extension should be given),
	\item \textsf{output1, output2, ...} --- paths to the output file(s).
\end{itemize}

For input there are additional parameters which can be set:
\begin{itemize}
	\item \textsf{-ci<value>} --- exclude $k$-mers occurring less than <value> times,
	\item \textsf{-cx<value>} --- exclude $k$-mers occurring more of than <value> times.
\end{itemize}

If additional parameters are not given they are taken from the appropriate input database. \\

Valid values for oper1, oper2,... are:
\begin{itemize}
	\item \textsf{sort} --- converts database produced by KMC2.x to KMC1.x database format (which contains $k$-mers in sorted order),
	\item \textsf{reduce} --- exclude too rare and too frequent $k$-mers,
	\item \textsf{compact} --- remove counters of $k$-mers,
	\item \textsf{histogram} --- produce histogram of $k$-mers occurrences,
	\item \textsf{dump} --- produce text dump of \textsf{KMC} database.	
\end{itemize}

For \textsf{sort}, \textsf{reduce} and \textsf{dump} operations additional \textsf{output\_params} are available:
\begin{itemize}
	\item \textsf{-ci<value>} --- exclude $k$-mers occurring less than <value> times,
	\item \textsf{-cx<value>} --- exclude $k$-mers occurring more than <value> times,
	\item \textsf{-cs<value>} --- maximal value of a counter.
\end{itemize}

If these parameters are not specified they are deduced based on input database. \\

For \textsf{histogram} operation additional \textsf{output\_params} are available:
\begin{itemize}
	\item \textsf{-ci<vaule>} --- minimum value of a counter to be stored in the output file (default value is a cutoff min stored in the database),
	\item \textsf{-cx<value>} --- maximum value of a counter to be stored in the output file (default value is a minimum of tree: $10^4$, cutoff max stored in the database, $2^{8\mathrm{CS}}-1$, where CS is the number of bytes used to store counters in the database)
\end{itemize}

For dump operation there are additional \textsf{oper\_params}:
\begin{itemize}
	\item \textsf{-s} --- force sorted output (default: false). \\
	For \textsf{KMC1.x} this parameter is irrelevant as $k$-mers are stored in sorted order and this order will be preserved in produced text file. For \textsf{KMC2.x} when this parameter is set $k$-mers will be sorted before dumpping to the text file.
\end{itemize}


\subsection *{example 1 - split $k$-mers on valid and invalid}
Let's suppose $k$-mers with occurrences below 11 are erroneous due to sequencing errors. With \textsf{reduce} we can split $k$-mer set to one set with valid $k$-mers and one with invalid: \\
kmc\_tools transform kmers reduce valid\_kmers -ci11 reduce erroneous\_kmers -cx10

\subsection*{example 2 - perform all operations}
kmc\_tools transform kmers reduce -ci10 reduced sort sorted compact without\_counters histogram histo.txt dump kmers.txt


\clearpage
\section{filter}
\label{sec:filter_operation}

This operation works with input FASTQ/FASTA files and a database produced by \textsf{KMC}.
It removes from the input read set those reads which does not contain specified number of $k$-mers in the input \textsf{KMC} database.
Currently, read names are completely ignored by kmc\_tools (though it may change in the future).

Syntax: \\
kmc\_tools [global\_params] filter [filter\_params] <kmc\_input\_db> [kmc\_input\_db\_params] <input\_read\_set> [input\_read\_set\_params]  <output\_read\_set> [output\_read\_set\_params] \\

where:

\begin{itemize}
	\item \textsf{kmc\_input\_db} --- path to database generated by \textsf{KMC},
	\item \textsf{input\_read\_set} --- path to input set of reads,
	\item \textsf{output\_read\_set} --- path to set output of reads.
\end{itemize}

filter\_params are:
\begin{itemize}
	\item \textsf{-t} --- trim reads on first invalid $k$-mer instead of remove entirely.
\end{itemize}

For $k$-mer database there are additional parameters:

\begin{itemize}
	\item \textsf{-ci$<$value$>$} --- exclude $k$-mers occurring less than <value> times,
	\item \textsf{-cx$<$value$>$} --- exclude $k$-mers occurring more of than <value> times.
\end{itemize}

For the input set of reads there are additional parameters:

\begin{itemize}
	\item \textsf{-ci$<$value$>$} --- remove reads containing less $k$-mers than value (but if -t is set the read is trimmed on first $k$-mer with counter lower than value),
	\item \textsf{-cx$<$value$>$} --- remove reads containing more $k$-mers than value (but if -t is set the read is trimmed on first $k$-mer with counter higher than value),
	\item \textsf{-f$<$a/q$>$} --- input in FASTA format (-fa), FASTQ format (-fq); default: FASTQ.
\end{itemize}
For input set of reads integer or floating number can be given as \textsf{-ci$<$value$>$} and \textsf{-cx$<$value$>$}. Integer values are used to define strict thresholds, which means only reads that contain at least $ci_{value}$ and at most $cx_{value}$ $k$-mers will be kept in the output read set.
Floating numbers for \textsf{-ci$<$value$>$} and \textsf{-cx$<$value$>$} parameters are used to define thresholds depending on read length. It should be in the range of [0.0;1.0]. Let $r$ be a length of a read. The read will be kept in the output read set only if it contains at least $\lfloor (r-k+1) * ci_{value} \rfloor$ and at most $\lfloor (r-k+1) * cx_{value} \rfloor$ $k$-mers which are present in \textsf{KMC} database. \\ \\
For the output set of reads there are additional parameters:
\begin{itemize}
	\item \textsf{-f$<$a/q$>$} --- output in FASTA format (-fa), FASTQ format (-fq); default: same as the input
\end{itemize}
\textsf{input\_read\_set} may be a single file or a file which contains a list of input files (one file per line). 

\section*{example}
 kmc\_tools filter kmc\_db -ci3 input.fastq -ci0.5 -cx1.0 filtered.fastq \\
 kmc\_tools filter kmc\_db input.fastq -ci10 -cx100 filtered.fastq \\
 kmc\_tools filter kmc\_db @input\_files.txt -ci10 -cx100 filtered.fastq



\end{document}
